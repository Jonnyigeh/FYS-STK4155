\documentclass[../main.tex]{subfiles}

\begin{document}

\section{Introduction}
The rapid rise of machine learning and AI in modern science has paved the way for numerous new technologies that us humans would not even think possible in the past. This new science has infiltrated every single field in science, be it physics, psychology or agriculture - each and every one of these fields find great use of machine learning and AI. 
\\ \\ As this field grows ever faster, the need for new minds to understand, develop and apply these methods is obvious. Which is exactly what we are looking to do in this scientific report. Here we will discuss, and investigate on of the simplest machine learning methods - namely that of linear regression. We will build our program bit by bit, starting simple and extending the complexity of our program as we delve deeper into linear regression. 
 \\ \\ In this report we will attempt to make a linear polynomial model in order to model terrain data. We will start by building our program with dummy data produced by the Franke-Function \footnote{See: \url{https://www.sfu.ca/~ssurjano/franke2d.html}}. We will make a polynomial fit to this function using different means of linear regression and resampling techniques, and assess the performance of each model using various error estimates.
 \\\\
To build our models we will use the methods of Ordinary Least Squares (OLS), Ridge Regression and LASSO, and resample our data using both bootstrapping and cross-validation. \\ \indent The necessary theory needed to understand the steps done in this report, as well as how the various method work - and how they've been implemented - will be presented neatly in the method and theory section. Lastly, the results from our analysis will be presented and discussed in the result and discussion section respectively.

\end{document}